\documentclass[12pt,a4paper]{article}%
\usepackage[T1]{fontenc}%
\usepackage[utf8]{inputenc}%
\usepackage{lmodern}%
\usepackage{textcomp}%
\usepackage{lastpage}%
\usepackage{ulem}%
\usepackage[HTML]{xcolor}%
\usepackage{soul}%
\usepackage{ifthen}%
\usepackage{fancyhdr}%
\usepackage{tikz}%
\usepackage{unicode-math}%
\usepackage{fontspec}%
\usepackage{ragged2e}%
%
\usepackage{geometry}%
\geometry{top=49mm, bottom=25mm, outer=10mm, inner=24mm}%
\usepackage{graphicx}%
\usepackage{array}%
\usepackage{eso-pic}%
\usepackage{tcolorbox}%
\usepackage{inputenc}%
\usepackage{enumitem}%
\setlist{topsep=0pt} %

            \pagestyle{fancy}
            \fancyhf{} % Clear existing headers and footers
            \fancyhead{} % Clear the header
            \fancyfoot[C]{% Centered footer
                \mbox{\tikz[baseline=(page.base)]{
                    \node[draw=black, fill=white, rounded corners=3pt, inner sep=3pt] (page) {\thepage};
                }}
            }
            \renewcommand{\headrulewidth}{0pt} % Remove the top black line
            \renewcommand{\footrulewidth}{0pt} % Remove any footer line
            %

    \newcommand\OddBackgroundPic{
        \put(0,0){%
            \parbox[b][\paperheight]{\paperwidth}{%
                \vfill
                \centering
                \includegraphics[width=\paperwidth,height=\paperheight]{/home/mac/boltStation/pythonBackend/Templates/1withHeader.jpg}
                \vfill
            }
        }
    }
    %

    \newcommand\EvenBackgroundPic{
        \put(0,0){%
            \parbox[b][\paperheight]{\paperwidth}{%
                \vfill
                \centering
                \includegraphics[width=\paperwidth,height=\paperheight]{/home/mac/boltStation/pythonBackend/Templates/1noHeader.jpg}
                \vfill
            }
        }
    }
    %
\renewcommand{\arraystretch}{1.5}%

    \AddToShipoutPictureBG{%
        \ifthenelse{\isodd{\value{page}}}{\OddBackgroundPic}{\EvenBackgroundPic}%
    }
    %
\definecolor{main_title_background_color}{HTML}{e2e8f0}%
\definecolor{main_title_border_color}{HTML}{1e293b}%
\definecolor{kinetic_base_color}{HTML}{f1f5f9}%
\definecolor{hypokinetic_color}{HTML}{fecdd3}%
\definecolor{akinetic_color}{HTML}{f87171}%
\definecolor{dyskinetic_color}{HTML}{fcd34d}%
%
\begin{document}%
\normalsize%
\begin{minipage}{0.5\linewidth}%
\textbf{\underline{Nom :}} \hspace{1cm} Zighoud%
\\%
\textbf{\underline{Prénom :}} \hspace{1cm} Sabine%
\\%
\end{minipage}%
\begin{minipage}{0.5\linewidth}%
\textbf{\underline{DDN :}} \hspace{1cm} 26-08-1991%
\\%
\textbf{\underline{Date :}} \hspace{1cm} 05/01/2025%
\\%
\end{minipage}%
\hspace{\textwidth}%
\\%
\begin{center}%

        \begin{tcolorbox}[
            colframe=main_title_border_color,        % Couleur de la bordure
            colback=main_title_background_color,        % Couleur de fond
            coltitle=main_title_border_color,       % Couleur du texte (si titre utilisé)
            arc=8pt,              % Rayon des coins arrondis
            boxrule=0.5mm,          % Épaisseur de la bordure
            auto outer arc,       % Ajuste automatiquement les coins arrondis
            width=\linewidth,     % Largeur de la boîte
            halign=center         % Centrer le texte horizontalement
        ]
        \LARGE{\textbf{ECG}}
        \end{tcolorbox}
        %
\end{center}%
%
\vspace*{\baselineskip}%
%
\vspace*{\baselineskip}%
\subsection*{Analyse du rythme:}%
\label{subsec:Analysedurythme}%

%
Rythme sinusal, avec une fréquence cardiaque de 85 bpm.  Le rythme est régulier, avec une relation onde P{-}QRS normale.%
\subsection*{Espace PR}%
\label{subsec:EspacePR}%

%
L’espace PR est allongé à 260 ms.%
\subsection*{Auriculogramme:}%
\label{subsec:Auriculogramme}%

%
Les ondes P sont normales, sans anomalies visibles.%
\subsection*{Ventriculogramme}%
\label{subsec:Ventriculogramme}%

%
Le complexe QRS présente un BBGc (bloc de branche gauche complet) avec une durée élargie à 160 ms.%
\subsection*{Repolarisation}%
\label{subsec:Repolarisation}%

%
Un sous{-}décalage du segment ST est observé en V2{-}V6, évoquant une lésion sous{-}épicardique, accompagné d’ondes T négatives en DII et aVF, compatibles avec une ischémie sous{-}épicardique.%
\subsection*{Segment QT}%
\label{subsec:SegmentQT}%

%
Le QT est mesuré à 450 ms. La fréquence cardiaque est de 85 bpm, le QTc corrigé calculé selon la formule de Bazett est de 462 ms. Absence d’anomalies du segment QT.%
\\%
%
\vspace*{\baselineskip}%
Remarques:%
\begin{itemize}%
\setlength{\itemsep}{0pt}%
\item%
Un sous{-}décalage ST en V2{-}V6 et ondes T négatives en DII et aVF, évoquant une ischémie sous{-}épicardique.%
\end{itemize}%
%
\vspace*{\baselineskip}%
\end{document}